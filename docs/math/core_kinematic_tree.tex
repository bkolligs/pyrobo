The kinematic tree is an abstract representation of a collection of transforms. 
Unlike the rigid collection (section \ref{sec: rigid}), the kinematic tree is able to represent transforms in multiple datum frames. 
The tree can then use the network of frames to "lookup" a transform between any two connected nodes, as hinted at in figure \ref{fig: poses}.
This concept is widely used in robotics and therefore has a place in the prototyping package. 
There are several algorithms this class implements:
\begin{enumerate}
	\item Representation: Construct a tree representation given only edges
	\item Lookup: apply breadth first graph search to find a path between any two nodes on the tree
	\item Root: express all frames in the base link frame or another specified frame on the tree
\end{enumerate}

\subsection{Representation}
Frames in a kinematic tree can have a child frame, a parent frame, or both. 
Frame $T$ is described as:
\begin{equation}
	T_i^{p(i)} = \begin{bmatrix}
		R_i^{p(i)} & t_i^{p(i)} \\
		\bf{0} & 1
	\end{bmatrix}
\end{equation}
where $i$ is the child frame, and $p(i)$ is the parent frame. 
The kinematic tree has a root, which we can also call the "base link". 